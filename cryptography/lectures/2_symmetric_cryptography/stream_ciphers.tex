\ifdefined\pres
\documentclass{beamer}
\else
\documentclass{paper}
\fi
\usepackage[utf8]{inputenc}
\usepackage[T2A,T1]{fontenc}
\usepackage[english, russian]{babel}
\usepackage{etoolbox}
\usepackage{amssymb}
\usetheme{Pittsburgh}

\selectlanguage{russian}
\newcommand{\define}[2]{{\bf #1} --- #2.\vspace{1em}}
\newcommand{\longdef}[1]{{\textbf{\underline{Опр:}} #1}}
\newcommand{\set}[1]{{\lbrace #1 \rbrace}}

\newtoggle{mynotes}

\ifdefined\pres
\togglefalse{mynotes}
\else
\toggletrue{mynotes}
\fi

\iftoggle{mynotes}{
  \newcommand{\mynote}[1]{mynote: #1}
}{
  \newcommand{\mynote}[1]{}
}

\title{Симметричная криптография. Потоковые шифры}
\author{Егор Никуленков}
\institute{ВГУ}
\date{2013}
\begin{document}

\frame{\titlepage}

\begin{frame}
  \frametitle{Свойства операции xor}

  \begin{itemize}
    \item{$a \oplus (b \oplus c) = (a \oplus b) \oplus c$}
    \item{$a \oplus a = 0$}
    \item{$a \oplus b = c \Rightarrow a \oplus c = b$}
  \end{itemize}
\end{frame}


% TODO: Remind where we have stopped. Show diagram of cryptographic algorithms (symmetric cryptography, public-key crypto).
% Show the place of stream ciphers in this diagram.
\begin{frame}
  \frametitle{Симметричная криптография}

  \begin{block}{Протокол передачи зашифрованного сообщения}
  \begin{enumerate}
    \item{Алиса и Боб выбирают систему шифрования}
    \item{Алиса и Боб выбирают ключ}
    \item{Алиса шифрует открытый текст с использованием алгоритма шифрования и ключа}
    \item{Алиса посылает шифрованное сообщение Бобу}
    \item{Боб дешифрует шифротекст с использованием того же алгоритма и ключа}
  \end{enumerate}
  \end{block}
\end{frame}


\begin{frame}
  \frametitle{Определение шифра}

  \longdef{Шифром, определенным на $(K,M,C)$, называется пара <<эффективно>> вычислимых алгоритмов $(E,D)$, где \newline
    \[E: K \times M \longrightarrow C \]
    \[D: K \times C \longrightarrow M, \]
    \[s.t. ~ \forall m \in M, k \in K: D(k, E(k,m)) = m \]
  }

  \mynote{
  $E$ часто является рандомизированным алгоритмом, т.е. использует
  некоторую случайную информацию во время своей работы. То есть, для одних
  и тех же значений  $k$ и $m$ результат $c$ может отличаться.
  Алгоритм же $D$ всегда детерминирванный. Для одних и тех же значений
  входных параметров $c$ и $k$ результат одинаков.
  }
\end{frame}


\begin{frame}
  \frametitle{Шифрование с помощью одноразовых блокнотов (one time pad)}

  \begin{itemize}
    \itemsep 2em
    \item{Множества, на которых определен шифр: \newline
      $M=C=\set{0,1}^{n}, K=\set{0,1}^{n}$}
    \item{Функция шифрования: \newline
      $E(k,m)=k \oplus m$}
    \item{Функция дешифрования: \newline
      $D(k,c)=k \oplus c$}

    \mynote{Пример работы алгоритма. Проверка равенства $D(k, E(k,m)) = m$}
    \mynote{Необходимо заметить, что алгоритм выполняется над потоком бит потенциально бесконечной длины}
    \mynote{По известному открытому сообщению и шифротексту можно легко найти
    часть ключа. Но эту часть ключа можно использовать только для расшифрования
    уже извесного открытого текста}
  \end{itemize}

\end{frame}


\begin{frame}
  \frametitle{Шифрование с помощью одноразовых блокнотов (one time pad)}

  \begin{block} {Преимущества}
    \begin{itemize}
      \item{Высокая скорость шифрования и дешифрования}
    \end{itemize}
  \end{block}

  \begin{block} {Недостатки}
    \begin{itemize}
      \item{Размер ключа равен размеру шифруемого текста}
    \end{itemize}
  \end{block}

  \vspace{2 em}
  Насколько хорош алгоритм с точки зрения безопасности?

\end{frame}


\begin{frame}
  \frametitle{Совершенная безопасность шифра}

  \textbf{Основная идея:} По известному шифротексту невозможно извлечь какую-либо информацию об открытом тексте.
  \mynote{Здесь предполагается, что злоумышленник может использовать только атаку
  на шифротекст, то есть по известному шифротексту пытаться узнать ключ или
  содержание открытого сообщения}
  \vspace{2em}

  \longdef{Шифр $(E,D)$, определенный на $(K,M,C)$ имеет совершенную безопасность, если \newline
    $\forall m_{0},m_{1} \in M:  |m_{0}| = |m_{1}|$ и $\forall c \in C$
      \begin{center} $Pr[E(k,m_{0}) = c] = Pr[E(k,m_{1}) = c]$, \end{center}
      где $k \stackrel{R}{\longleftarrow}K$}

  \mynote{Пояснение: Если известно некоторое зашифрованное сообщение $c$, то
    вероятности того, что зашифровано сообщение $m_{0}$ или $m_{1}$ равны. То
    есть по зашифрованному сообщению невозможно определить какое из сообщений
    более вероятно было зашифровано}
  \mynote{Шеннон теоретически показал, что совершенная безопасность возможна}

\end{frame}


\begin{frame}
  \frametitle{Совершенная безопасность шифра}

  \begin{itemize}
    \item{Шифрование с помощью одноразовых блокнотов имеет совершенную безопасность}
    \item{Для того, чтобы шифр имел совершенную безопасность, необходимо, чтобы $|K| \ge |M|$, поэтому использование
          шифров с совершенной безопасностью на практике затруднено}
  \end{itemize}
\end{frame}


\begin{frame}
  \frametitle{Потоковые шифры}

  \textbf{Основная идея:} Замена "случайного" ключа на "псевдослучайный" ключ.
  \vspace{1em}

  Для этого используются генераторы псевдослучайных чисел (ГПЧ).
  \vspace{1em}

  \[G:\set{0,1}^{s} \mapsto \set{0,1}^{n}, n \gg s\]

  \mynote{Требования к генератору: 
    1. Должен существовать детерминистический алгоритм, который может быть эффективно выполнен для реализации генератора.
       Единственный элемент, который остается случайным - это seed.
    2. Выходная строка должна выглядеть случайно
    Открытый вопрос: что значит выглядеть случайно?
  }

\end{frame}


\begin{frame}
  \frametitle{Потоковые шифры}

  Так как ГПЧ может генерировать псевдослучайные строки большой длины, то можно создать шифр,
  аналогичный шифру с одноразовыми блокнотами. Главное отличие - длина ключа фиксирована и много меньше, чем длина
  сообщения.

  \begin{itemize}
    \itemsep 2em
    \item{Функция шифрования: \newline
      $E(k,m)=m \oplus G(k)$}
    \item{Функция дешифрования: \newline
      $D(k,c)=c \oplus G(k)$}

    \mynote{Показать на рисунке как из маленького ключа получается длинная последовательность}
    \mynote{Открытый вопрос: насколько такой шифр безопасен? Необходимо новое определение безопасности}
  \end{itemize}

\end{frame}


\begin{frame}
  \frametitle{Потоковые шифры}

  \mynote{Главным уязвимым звеном в потоковых шифрах становится ГСЧ}

  Необходимое условие безопасности поточного шифра - непредсказуемость ГСЧ.

  \vspace{1em}

  Если ГСЧ предсказуем, то

  \[\exists i: G(k)|_{1 \ldots i} \stackrel{alg}{\longrightarrow} G(k)|_{i+1 \ldots n}\]

  \mynote{Предсказуемость ГСЧ может полностью свести безопасность потокового шифра на нет.
    Пример - протокол SMTP. Так как любое письмо начинается с FROM:, то злоумышленник может определить первые 5
    байт, сгенерированных ГСЧ и предсказать последующие биты ГСЧ}

  \mynote{Многие реализации ГСЧ, которые поставляются в стандартных библиотеках, на самом деле легко предсказуемы.
    Например, реализация функции random() в библиотеке glibc. Kerberos v.4 использовало эту функцию, что привело к её взлому.}

\end{frame}


\begin{frame}
  \frametitle{Предсказуемость ГСЧ}

  \longdef{ГСЧ $G:K \longrightarrow \set{0,1}^{r}$ называется предсказуемым, если существует
   эффективно вычислимый алгоритм $A$ и $\exists 1 \le i \le n-1$ такое, что
    \begin{displaymath}
      {Pr_{k \stackrel{R}{\longleftarrow} K}[A(G(k))|_{1 \ldots i} = G(k)|_{i+1}] \ge 1/2 + \varepsilon}
    \end{displaymath}
  }

\end{frame}


\begin{frame}
  \frametitle{Предсказуемые ГСЧ}

  \begin{block}{Линейный конгруэнтный метод}
    \begin{itemize}
      \item{Три параметра: a, b, p}
      \item{$r[0] = seed$}
      \item{$r[i] = (a*r[i-1]+b)~mod~p$}
      \item{Используется в чистом виде или с некоторыми модификациями во многих реализациях стандартных языковых библиотеках}
    \end{itemize}
  \end{block}

\end{frame}


\begin{frame}
  \frametitle{Потоковые шифры. Атаки}

  \begin{block}{Повторное использование ключа в шифровании одноразовыми блокнотами}
    \begin{itemize}
      \item{Ключ для поточного шифра не должен использоваться больше одного раза
        \[ c_{1} \leftarrow m_{1} \oplus PRG(k) \]
        \[ c_{2} \leftarrow m_{2} \oplus PRG(k) \]
        \[ c_{1} \oplus c_{2} =  m_{1} \oplus m_{2} \]
      }
      \item{Существующая избыточность человеческого языка и ASCII-кодирования позволяет восстановить открытый текст}
      \item{Примеры повторного использования ключа: проект "Венона", MS-PPTP, WEP}
      \item{Невозможность использования потоковых шифров в некоторых областях (шифрование данных на диске)}
    \end{itemize}
  \end{block}

\end{frame}


\begin{frame}
  \frametitle{Потоковые шифры. Атаки}

  \begin{block}{Невозможность проверки целостности}
    Злоумышленник может изменить зашифрованное сообщение предсказуемым образом:
      \[ c_{Alice} = m \oplus k \]
      \[ c_{Mallory} = c_{Alice} \oplus p \]
      \[ m_{Bob} = c_{Mallory} \oplus k = m \oplus p \]
  \end{block}

\end{frame}


% TODO: Describe existing algorithms in more detail
\begin{frame}
  \frametitle{Потоковые шифры. Примеры}

  \begin{itemize}
    \item{RC4}
    \item{Шифры, использующие регистр сдвига с линейной обратной связью (LFSR)}
    \item{Семейство шифров eStream}
  \end{itemize}

\end{frame}

\end{document}
