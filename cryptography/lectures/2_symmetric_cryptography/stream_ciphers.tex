\documentclass{beamer}
\usepackage[utf8]{inputenc}
\usepackage[T2A,T1]{fontenc}
\usepackage[english, russian]{babel}
\usetheme{Pittsburgh}

\selectlanguage{russian}
\newcommand{\define}[2]{{\bf #1} --- #2.\vspace{1em}}
\newcommand{\longdef}[1]{{\textbf{\underline{Опр:}} #1}}
\newcommand{\set}[1]{{\lbrace #1 \rbrace}}

\title{Симметричная криптография. Потоковые шифры}
\author{Егор Никуленков}
\institute{ВГУ}
\date{2013}
\begin{document}

\frame{\titlepage}

\begin{frame}
  \frametitle{Свойства операции xor}

  \begin{itemize}
    \item{$a \oplus (b \oplus c) = (a \oplus b) \oplus c$}
    \item{$a \oplus a = 0$}
    \item{$a \oplus b = c \Rightarrow a \oplus c = b$}
  \end{itemize}
\end{frame}


% Remind where we have stopped. Show diagram of cryptographic algorithms (symmetric cryptography, public-key crypto).
% Show the place of stream ciphers in this diagram.

\begin{frame}
  \frametitle{Шифрование с помощью одноразовых блокнотов (one time pad)}

  \begin{itemize}
    \itemsep 2em
    \item{Функция шифрования: \newline
      $E(k,m)=k \oplus m$}
    \item{Функция дешифрования: \newline
      $D(k,c)=k \oplus c$}

    \note{Пример работы алгоритма. Проверка равенства $D(k, E(k,m)) = m$}
    \note{Необходимо заметить, что алгоритм выполняется над потоком бит потенциально бесконечной длины}
    \note{По известному открытому сообщению и шифротексту можно можно легко найти ключ}
  \end{itemize}

\end{frame}


\begin{frame}
  \frametitle{Шифрование с помощью одноразовых блокнотов (one time pad)}

  \begin{block} {Преимущества}
    \begin{itemize}
      \item{Высокая скорость шифрования и дешифрования}
    \end{itemize}
  \end{block}

  \begin{block} {Недостатки}
    \begin{itemize}
      \item{Размер ключа равен размеру шифруемого текста}
    \end{itemize}
  \end{block}

  \vspace{2 em}
  Насколько хорош алгоритм с точки зрения безопасности?

\end{frame}


\begin{frame}
  \frametitle{Совершенная безопасность шифра}

  \textbf{Основная идея:} По известному шифротексту невозможно извлечь какую-либо информацию об открытом тексте.
  \vspace{2em}

  \longdef{Шифр $(E,D)$, определенный на $(K,M,C)$ имеет совершенную безопасность, если \newline
    $\forall m_{0},m_{1} \in M:  |m_{0}| = |m_{1}|$ и $\forall c \in C$
      \begin{center} $Pr[E(k,m_{0}) = c] = Pr[E(k,m_{1}) = c]$, \end{center}
      где $k \stackrel{R}{\longleftarrow}K$}

  \note{Пояснение: Если известно некоторое зашифрованное сообщение $c$, то вероятности того, что зашифровано
        сообщение $m_{0}$ или $m_{1}$ равны. То есть по зашифрованному сообщению невозможно определить какое из сообщений
        более вероятно было зашифровано}
  \note{Шеннон теоретически показал, что совершенная безопасность возможна}

\end{frame}


\begin{frame}
  \frametitle{Совершенная безопасность шифра}

  \begin{itemize}
    \item{Шифрование с помощью одноразовых блокнотов имеет совершенную безопасность}
    \item{Для того, чтобы шифр имел совершенную безопасность, необходимо, чтобы $|K| \ge |M|$, поэтому использование
          шифров с совершенной безопасностью на практике затруднено}
  \end{itemize}
\end{frame}


\begin{frame}
  \frametitle{Потоковые шифры}

  \textbf{Основная идея:} Замена "случайного" ключа на "псевдослучайный" ключ.
  \vspace{1em}

  Для этого используются генераторы псевдослучайных чисел (ГПЧ).
  \vspace{1em}

  %TODO: replace >> with better sign
  \[G:\set{0,1}^{s} \mapsto \set{0,1}^{n}, n>>s\]

  \note {Требования к генератору: 
    1. Должен существовать детерминистический алгоритм, который может быть эффективно выполнен для реализации генератора.
       Единственный элемент, который остается случайным - это seed.
    2. Выходная строка должна выглядеть случайно
    Открытый вопрос: что значит выглядеть случайно?
  }

\end{frame}


\begin{frame}
  \frametitle{Потоковые шифры}

  Так как ГПЧ может генерировать псевдослучайные строки большой длины, то можно создать шифр,
  аналогичный шифру с одноразовыми блокнотами. Главное отличие - длина ключа фиксирована и много меньше, чем длина
  сообщения.

  \begin{itemize}
    \itemsep 2em
    \item{Функция шифрования: \newline
      $E(k,m)=m \oplus G(k)$}
    \item{Функция дешифрования: \newline
      $D(k,c)=c \oplus G(k)$}

    \note{Показать на рисунке как из маленького ключа получается длинная последовательность}
    \note{Открытый вопрос: насколько такой шифр безопасен? Необходимо новое определение безопасности}
  \end{itemize}

\end{frame}


\begin{frame}
  \frametitle{Потоковые шифры}

  \note{Главным уязвимым звеном в потоковых шифрах становится ГСЧ}

  Необходимое условие безопасности поточного шифра - непредсказуемость ГСЧ.

  \vspace{1em}

  Если ГСЧ предсказуем, то

  \[\exists i: G(k)|_{1,..,i} \stackrel{alg}{\longrightarrow} G(k)|_{i+1,...,n}\]

  \note{Предсказуемость ГСЧ может полностью свести безопасность потокового шифра на нет.
    Пример - протокол SMTP. Так как любое письмо начинается с FROM:, то злоумышленник может определить первые 5
    бит, сгенерированные ГСЧ и предсказать последующие биты ГСЧ}

  \note{Многие реализации ГСЧ, которые поставляются в стандартных библиотеках, на самом деле легко предсказуемы.
    Например, реализация функции random() в библиотеке glibc. Kerberos v.4 использовало эту функцию, что привело к её взлому.}

\end{frame}

\end{document}
